\documentclass{article}
\usepackage{url}
\begin{document}

\title{Hatchet \\ 
A Type Checking and Inference Tool for Haskell 98\\
Version 0.1}
\author{Bernie Pope, bjpop@cs.mu.oz.au}
\date{2002}

\maketitle

\section{Introduction}

Hatchet is a type checking and inference tool for 
Haskell 98, written in (almost) 
Haskell 98.\footnote{Hereafter referred to as Haskell.}
This document is an attempt to describe what Hatchet is, 
what it is not, and briefly how it works.

\section{Manifesto}

Types are fundamental in Haskell. If you want to manipulate
Haskell programs (and fragments) then you must be able to
find and check the types from the text of the code. Hatchet is
intended to be a tool for that very purpose.

Haskell has a static type system. This means that types 
are dealt with completely at compile time; consequently
there is no type checking during program execution. Haskell's
type system has foundations in the Hindley Milner system, 
though it involves type-checking as well as inference and it is
extended with a system of constraints for handling 
type classes. 

The authoritative definition of Haskell is the
Language Report. Unfortunately the Report only gives an
informal treatment of the type system. How then does one reason
about the types in their programs? The usual approach is to
ask a compiler to compile the program. If there are any type
errors the compiler will hopefully report them. If there
are no type errors, then one might ask for the compiler to
report the types of certain entities. 

The problem with this approach is that compilers are complicated systems 
that have to deal with more than just the static properties
of programs. It was found to be quite difficult to remove the
type checkers from the front of any of the existing Haskell
implementations. Furthermore, the type systems of all 
existing implementations are different, particularly with respect
to extensions of the language beyond what is prescribed in the
Report.

Mark Jones provides one possible interpretation of the Haskell
type system in his work \emph{Typing Haskell in Haskell}. 
The kernel of a typing system for Haskell written in Haskell
accompanies that work. Hatchet is based on that kernel.

Hatchet is supposed to be as simple as possible, particularly
so that it can be understood and used by mortals. Simplicity
comes at a price. Hatchet is not particularly fast and it doesn't
have particularly good error messages (for the moment).
It is not really designed for experimentation with type systems;
it might be possible to use it in that way, but that is not a
design goal. The target application is type directed 
program transformation and manipulation (in which you know you
have a type correct program, but you don't already know what all
the types are).

Lastly Hatchet is not in any way intended to be a definition of
the Haskell type system. Version 0.1 is certainly incomplete,
and most likely \emph{wrong} in some circumstances. Hopefully that 
future versions will be less wrong. 

\section{Building Hatchet from Source}

\begin{enumerate}
\item Download the source from the Hatchet web page: 
\begin{verbatim}
www.cs.mu.oz.au/~bjpop/hatchet.html
\end{verbatim}
\item Untar and unzip the source bundle into a directory,
\item \texttt{make depend}
\item \texttt{make}
\end{enumerate}

You will need ghc. By default this will create an executable called
\texttt{hatch}. It ought to work with nhc, but this is yet to be tested.
Hugs seems to choke at various points. 

\section{Usage}

When given the filename of a Haskell module, Hatchet will print the types
or all top-level and let/where bound variables in the module to the standard
output:
\begin{verbatim}
   ./hatch Foo.hs
\end{verbatim}
Alternative information must be requested more explicitly, such as:
\begin{verbatim}
   ./hatch -d kinds Foo.hs
\end{verbatim}
Hatchet has a number of other command line arguments. To see what is
available and what they do:
\begin{verbatim}
   ./hatch -h 
\end{verbatim}


\section{Features}

Hatchet can tell you a lot of information about your Haskell module.
In particular it can provide:

\begin{itemize}
\item the parse tree of the module, 
\item the source location and type of binding environment
      for every identifier,
\item the dependency graph of all top-level identifiers,
\item the kind of all type constructors and class parameters,
\item the class hierarchy, including superclass relationships,
      known instances, and method signatures,
\item the type of every data constructor, and
\item the type of every top-level and let/where bound identifier
\end{itemize}

\section{Missing Features}

Version 0.1 is intended as a hacker's release. There are a number of
missing features, the most significant are:

\begin{itemize}
\item full support for multi-module programs,
\item user defined and instantiated classes, and
\item record syntax 
\end{itemize}

Basically version 0.1 of hatchet can type single module 
programs that make use of the standard Prelude. Future versions
should rectify these shortcomings.

\section{Structure/Design}

In this section the design of Hatchet is described as a series 
of tasks that are performed (in order) and relate each task to
the most significant modules in the Hatchet source. This is
a rough guide, and for simplicity certain details are
not mentioned.

\begin{enumerate}
\item command line options 
     \begin{description}
     \item[modules] \texttt{Opts}, \texttt{GetOpt}
     \end{description}
\item lexing and parsing
     \begin{description}
     \item[modules] \texttt{HsLexer}, \texttt{HsParser.ly/hs}, 
                    \texttt{HsParseMonad}, \texttt{HsParseUtils}, 
                    \texttt{HsSyn}, \texttt{HsParsePostProcess}
     \item[notes] Due to peculiarities in the definition of Haskell,
                  the abstract syntax that results from the parser is 
                  not entirely accurate, and will require re-structuring
                  in some circumstances, this relates primarily to
                  infix applications and multi-equation functions.
     \end{description}
\item syntax conversion 
     \begin{description}
     \item[modules] \texttt{SynConvert}, \texttt{AnnotatedSyntax}
     \item[notes] The abstract syntax description
                  obtained after parsing is converted into a slightly more liberal
                  representation that allows qualified identifiers
                  everywhere - this simplifies renaming later. It would
                  be better to use just the one abstract syntax, rather
                  than two, but the maintainers of \texttt{HsSyn} have 
                  decided not to allow qualified names everywhere in
                  accordance with the Language Report. The resulting
                  \emph{annotated} syntax is therefore unlikely to remain
                  within the strict confines of Haskell 98. 
     \end{description}
\item initial type environment construction 
     \begin{description}
     \item[modules] \texttt{MultiModule}, \texttt{MultiModuleBasics},
                    \texttt{HaskellPrelude}
     \item[notes] This is not quite complete. The interface
                  files for any modules that the current one depends
                  upon are read. This presumes that they have been type-checked
                  previously. The topological ordering of modules is not
                  implemented by Hatchet and is presumed to be done
                  by someone else (perhaps hmake?). A type environment
                  is composed from the information in the interface files
                  and a default environment for the Prelude.
     \end{description}
\item module deconstruction 
     \begin{description}
     \item[modules] \texttt{TidyModule}
     \item[notes] The abstract syntax is broken into separate components.
                  For example, all the variable declarations are collected
                  together, and all the data declarations together, and 
                  so on. 
     \end{description}
\item desugaring 
     \begin{description}
     \item[modules] \texttt{Desugar}
     \item[notes] Pattern bindings are simplified, so for example:
\begin{verbatim}
(x,y,z) = foo
\end{verbatim}
                  becomes:
\begin{verbatim}
newRhs = foo
x = (\(a, _, _) -> a) newRhs
y = (\(_, a, _) -> a) newRhs
z = (\(_, _, a) -> a) newRhs
\end{verbatim}
                 Type synonyms are removed completely. Do notation is
                 converted into expression form. 
                 Expressions with type annotations are converted to
                 let expressions following the Language Report:
\begin{verbatim}
e :: t
\end{verbatim}
                  becomes:
\begin{verbatim}
let {v :: t, v = e} in v
\end{verbatim}
                 Desugaring tends to
                 simplify the task of type inference because it reduces
                 the number of cases that need to be considered. However,
                 it changes the structure of the program, and in some 
                 cases introduces new identifiers into the program. 
                 One design goal of Hatchet is to minimise the amount
                 of desugaring that is performed.
     \end{description}
\item renaming 
     \begin{description}
     \item[modules] \texttt{Rename}
     \item[notes] Every variable identifier (values and types) in the 
                  module is uniquely renamed and
                  qualified to its module of origin. This simplifies the
                treatment of variable scope and allows multiple variables
                of a given (original) name to exist in the type environment
                at the same time. This is particularly useful since we 
                collect the types of let and where bound identifiers
                as well as top-level identfiers and so we must be
                prepared for overlapping scopes.
                The renaming scheme is simple: a unique
                integer is prepended on the front of each occurrence of
                a given identifier. This facilitates the simple retrieval
                of the original name and also indicates which names have
                been renamed and which haven't (numerals are not allowed
                to start an identifier in Haskell, so any name with numerals
                at the front must have been renamed, otherwise it would
                have been rejected by the parser). Type constructors and
                data constructors and class names are not renamed 
                however they are qualified to their module of origin.
     \end{description}
\item infix patching 
     \begin{description}
     \item[modules] \texttt{Infix}
     \item[notes] The parsing of Haskell is complicated by the presence
                of infix rules. Correct parsing of an expression depends on the 
                associativity and binding power of any identifiers
                involved in the expression. The result is that you cannot
                always correctly parse a Haskell module without knowing
                the fixities of imported identifiers. Knowing the fixities
                would entail at least a partial parsing of imported modules.
                The parser used in Hatchet (\texttt{HsParser}) presumes
                that all infix applications can be parsed left-to-right and
                that infix applications might require later re-structuring.
                This seems like a simple approach. All infix rules from
                imported modules are collected with the infix rules for the
                current module and then the abstract syntax is re-structured
                based on those rules where infix applications occur.
     \end{description}
\item kind inference 
     \begin{description}
     \item[modules] \texttt{KindInference}, \texttt{DependAnalysis}, \texttt{Digraph}
     \item[notes] Kind inference is performed for all type constructors and classes.
                  Inference is not needed for type synonyms because by this stage they
                  have been removed. 
     \end{description}
\item data constructor types 
     \begin{description}
     \item[modules] \texttt{DataConsAssump}
     \item[notes] The type assumptions for all data constructors are collected. They
                  will be needed during type inference. Collecting types
                  of data constructors is straightforward.
     \end{description}
\item class hierarchy construction 
     \begin{description}
     \item[modules] \texttt{Class}
     \item[notes] This is incomplete. The idea is to update the class hierarchy
                  with new classes and instance declarations. The existing class
                  hierarchy comes from imported modules and the Prelude.
     \end{description}
\item dependency analysis 
     \begin{description}
     \item[modules] \texttt{DependAnalysis}, \texttt{DeclsDepends}, \texttt{Digraph}
     \item[notes] This is just dependency analysis for top-level variable declarations.
                  Further dependency analysis will be required when the types of local
                  declarations are computed. 
      \end{description}
\item type inference and checking 
     \begin{description}
     \item[modules] \texttt{TIMain}, \texttt{TIMonad}, \texttt{Representation},
     \texttt{Type}, \texttt{Class}
     \item[notes] Finally the types of all top-level and let/where bound
                  identifiers are computed. This is based on the type-checking
                  kernel from \emph{Typing Haskell in Haskell}. Types are 
                  collected in an environment which is threaded through the
                  computation. 
     \end{description}
\item interface file writing 
     \begin{description}
     \item[modules] \texttt{MultiModule}, \texttt{MultiModuleBasics}
     \item[notes] Module specific information about types, kinds, classes,
                 infix rules, and type synonyms
                  are written to an intermediate file. The default behaviour
                  is not to create an intermediate file unless the \texttt{-i}
                command is supplied. When an intermediate file is required for
                a module called \texttt{Foo}, the default intermediate filename
                is \texttt{Foo.ti}, however an alternative name is allowed
                as an argument to the \texttt{-i} flag. The writing of intermediate
                files is inefficient and incomplete. 
     \end{description}
\end{enumerate}

\section{Divergence from Typing Haskell in Haskell}

The typing algorithm in Mark Jones' \emph{Typing Haskell 
in Haskell} requires the program to be decomposed into 
groups of mutually recursive functions. Each group of such
functions is further divided into explicitly typed and
implicitly typed function bindings. It is possible to 
further sub-divide the implicitly typed functions into
smaller dependency groups. Hatchet does not do this
further sub-division. So for example, the following 
program can be typed according to the rules of 
\emph{Typing Haskell in Haskell} but not by Hatchet:

\begin{verbatim}
f :: a -> b
f x = g True
g y = h True
h z = f True
foo = h () 
\end{verbatim}

Hugs accepts this program, but ghc does not.

\section{Known Bugs}

\begin{itemize}
\item Modules which explcitly \texttt{import Prelude} will fail. A fix for
      this will come when multi-module programs are supported properly.
\item Locally bound variables (let and where) will sometimes have their
      types printed incorrectly:
\begin{verbatim}       
foo = True
    where
    x = 2
\end{verbatim}
      For the above program, Hatchet will report:
\begin{verbatim}       
Main.1_foo :: Bool
Main.2_x :: a
\end{verbatim}
      This is due to the way that constraints are handled for
      locally bound variables. 
\end{itemize}

\section{The future of Hatchet}

There is much work to be done. Who is going to do it? Hopefully
Hatchet will be useful to people in the Haskell community, and
hopefully it will be improved by those people who use it. 

Some code restructuring might take place if and when the Haskell library
hierarchies are standardised and implemented by compilers.

It would be nice to make Hatchet more like a library and less like
a stand-alone program. Unfortunately, Haskell is a difficult language to
deal with in fragments, it just doesn't look like it was desgined
with meta-programming in mind. 

It would be nice to try and apply some software engineering principles
to the development of Hatchet. In particular there are a lot of 
passes over the syntax tree. This code is tedious to write and maintain.
Perhaps something like \emph{Strafunski} can help? Any volunteers?

Hatchet also needs documentation and test cases (both positive and negative).
A small amount of testing has been done, but there will never be enough.
Bug reports and patches should be sent to Bernie \url{bjpop@cs.mu.oz.au},
who will do his best with limited time to fix and apply them.
If you are a keen Haskell programmer and have some spare time
please jump in and start improving Hatchet.

\section{License, copyright and credits}

Hatchet version 0.1 is derived from a number of sources
each with their own copyright and license restrictions.
See each file in the source distribution for more 
information.

Special license and copyright information for all work
derived directly and indirectly from \emph{Typing Haskell
in Haskell} by 
Mark Jones\footnote{\url{http://www.cse.ogi.edu/~mpj/thih/}}
is provided in the file \texttt{License.thih}. Please read
that file carefully and abide by its contents.

Hatchet is free software. Permission is granted to
all good life-forms and machines to use and distribute
it in whole or part, as source or binary, modified or
unmodified, for any purpose providing the Licenses
and Copyrights of the component sources are also
respected and followed.

The following people have contributed to Hatchet:
\begin{itemize}
\item Bernie Pope     
\item Bryn Humberstone
\item Toby Ord
\item Lindsay Powles
\item Robert Shelton
\end{itemize}

Hatchet has derived much benefit from the work of 
many others including:

\begin{itemize}
\item Mark Jones         (Typing Haskell in Haskell) 
\item Manuel Chakravarty (FiniteMaps) 
\item Sven Panne         (GetOpt) 
\item Graham Hutton      (ParseLib) 
\item Erik Meijer        (ParseLib) 
\item The GHC team       (Digraph, Haskell Parsing and Pretty Printing)
\end{itemize}

Thankyou to all those involved.

\section{References}

\begin{itemize}
\item Mark Jones, \emph{Typing Haskell in Haskell}, \url{http://www.cse.ogi.edu/~mpj/thih}
\item The Haskell 98 Language Report, \url{http://www.haskell.org/onlinereport}
\item The Glasgow Haskell Compiler, \url{http://haskell.cs.yale.edu/ghc}
\item Strafunski, \url{http://www.cs.vu.nl/Strafunski}
\end{itemize}

\end{document}
